\documentclass[12pt, a4paper, simple]{eskdtext}

\usepackage{hyperref}
\usepackage{env}
\usepackage{_sty/gpi_lst}
\usepackage{_sty/gpi_toc}
\usepackage{_sty/gpi_t}
\usepackage{_sty/gpi_p}
\usepackage{_sty/gpi_u}

% Код
\ESKDletter{}{К}{Р}
\def \gpiDocTypeNum {81}
\def \gpiDocVer {00}
\def \gpiCode {\ESKDtheLetterI\ESKDtheLetterII\ESKDtheLetterIII.\gpiStudentGroupName\gpiStudentGroupNum.\gpiStudentCard-0\gpiDocNum~\gpiDocTypeNum~\gpiDocVer}

\def \gpiDocTopic {ПОЯСНИТЕЛЬНАЯ ЗАПИСКА}

% Графа 1 (наименование изделия/документа)
\ESKDcolumnI {\ESKDfontIII \gpiTopic \\ \gpiDocTopic}

% Графа 2 (обозначение документа)
\ESKDsignature {\gpiCode}

% Графа 9 (наименование или различительный индекс предприятия) задает команда
\ESKDcolumnIX {\gpiDepartment}

% Графа 11 (фамилии лиц, подписывающих документ) задают команды
\ESKDcolumnXIfI {\gpiStudentSurname}
\ESKDcolumnXIfII {\gpiTeacherSurname}
\ESKDcolumnXIfV {\gpiTeacherSurname}

\begin{document}
    \input{_tex/gpi_pz_titlePage.tex}

    \ESKDthisStyle{empty}
    лист с заданием
    \newpage
    \ESKDthisStyle{formII}

    % Содержание
    \tableofcontents                                
    \paragraph{ПРИЛОЖЕНИЕ А. СХЕМА ПРОГРАММЫ}
    \paragraph{ПРИЛОЖЕНИЕ Б. ТЕКСТ ПРОГРАММЫ}
    \newpage

    %
    \newpage
    \addcontentsline{toc}{section}{ВВЕДЕНИЕ}
    \section*{ВВЕДЕНИЕ}
    \newpage

    %
    \section{АНАЛИЗ ПОСТАВКИ ЗАДАЧИ}
    \subsection{Перечень функций}

    Достижение цели курсовой работы предполагает необходимость создания веб- или мобильное приложение (ОС Android/iOS) со следующим функционалом:
    
    \begin{enumerate}
        \item[1.] Предусмотреть возможность игры с ИИ.
        \item[2.] Разработать фиксированный набор карт для игры (от 30 и более).
        \item[3.] Сохранение статистики матчей (опционально - сохранение на удаленном сервере).
        \item[4.] Поддержка сетевого матча один-на-один.
        \item[5.] Минимальный пользовательский интерфейс и графическое оформление карт.
    \end{enumerate}
    
    \subsection{Требования пользователей}

    Пользовательские требования - описание на естественном языке (плюс поясняющие диаграммы) функций, выполняемых системой, и ограничений, накладываемых на неё.

    Ответственный: Системный аналитик

    Эти требования должны определять только внешнее поведение системы, избегая по возможности определения структурных характеристик системы.
    Пользовательские требования должны быть написаны естественным языком с использованием простых таблиц, а также наглядных и понятных диаграмм.

    \textbf{Обязательные} требования игрока к игре:

    \begin{enumerate}
        \item [1.] Разработать фиксированный набор карт для игры (от 30 и более).
        \item [2.] Минимальный пользовательский интерфейс и графическое оформление карт.
    \end{enumerate}

    \textbf{Желательные} требования к игре:

    \begin{enumerate}
        \item [1.] Предусмотреть возможность игры с ИИ.
        \item [2.] Сохранение статистики матчей (опционально - сохранение на удаленном сервере).
        \item [3.] Поддержка сетевого матча один-на-один.
    \end{enumerate}

    \newpage
    \subsection{Описание предметной области}

    \textbf{Игрок} - человек или машина, который/которая играет в данную игру.
    В данной игре будут два игрока: мы и игрок противник.

    \textbf{Карты} - графический элемент игры, в виде прямоугольника с характеристиками.
    Карт в игре в количестве 30 штук. 30 штук как и у нас, так и у противника.
    Карты имеют фракции: мечник, лучник, тяжелая альтилерия; который сдавятся в определеное поле боя.
    
    \textbf{Мечник} - категория карты, которая имеет малый урон и сражается с мечниками противника.

    \textbf{Лучник} - категория карты, которая имеет малый урон и сражается с лучниками противника.

    \textbf{Тяжелая альтилерия} - катерия карты, которая имеет высокий урон и сражается с тяжелой альтилерией противника.

    \textbf{Поле боя} - игровое поле, которое разбито на 6 частей, из которых три части противника и три части наши.
    
    Поле боя содержит:

    \begin{enumerate}
        \item [1.] Тяжелую альтилерию противника.
        \item [2.] Лучников противника.
        \item [3.] Мечников противника.
        \item [4.] Своих мечников.
        \item [5.] Своих лучников.
        \item [6.] Свою тяжелую альтилерию.
    \end{enumerate}

    \textbf{ПАС} - действие, после объявления которого мы объявляем о прекращении бойни и пропускаем ход, в тоже время противнику дается последний выбор карты,
    после чего подсчитывается игровой счёт. В течении игры можна два раза нажать на кнопку <<ПАС>>.
    После подсчёта первого <<ПАС>> карты с поля боя идут на кладбище. Набираются карты до десяти.
    Происходит игра. После подсчёта второго <<ПАС>> сравниваются победы и объявляется победитель.

    \newpage
    \subsection{Варианты использования программы в виде диаграмм прецедентов (use case)}

    \begin{figure}[!h]
        \centering
        \includegraphics[]
            {_assets/UML_Use_case_diagram.png}
        \caption{Первичное описание прецедентов}
    \end{figure}

    \paragraph{}\hspace{0pt}

    \textbf{\underline{Прецендент №1 <<Выбор карты>>}}

    \textbf{Назначение}: из своей колоды карт выбираем карту, которую ходим выставить на поле.

    \textbf{Исполнители}: игрок.

    \textbf{Предусловие}: есть карта/карты, которую/которые можно выбрать.

    \textbf{Основной поток событий}: игрок выберет карту и карты попадет на поле (мечника, лучника, тяжелой альтилерии),
    иначе АПС.

    \textbf{Альтернативный поток событий}: если у игрока не будет карт, то игрок пропускает ход;
    если игрок не выбирает карту, то по истечению времени карта выбирается автоматически.

    \paragraph{}\hspace{0pt}

    \textbf{\underline{Прецендент №2 <<Завершения сражения (ПАС)>>}}

    \textbf{Назначение}: завершить бой и узнать результат бойни.

    \textbf{Исполнители}: игрок.

    \textbf{Предусловие}: есть карты на поле.

    \textbf{Основной поток событий}: мы пропускаем ход, а противник выбирает последнюю карту, после чего проиходит подчёт сил, иначе АПС.

    \textbf{Альтернативный поток событий}: игрок может продолжать выставлять карты на поле.

    \subsection{Первичное описание объектов и классов, прецедентов системы}
    \subsection{Первоначаьное описание отношений между классами}
    \subsection{Диаграмма состояний (statechart diagram) для прецендентов}
    \newpage

    %
    \section{ПРОЕКТИРОВАНИЕ СТРУКТУРЫ ПРИЛОЖЕНИЯ}
    \subsection{Диаграммы классов предметной области}
    \subsection{Графический интерфейс приложения}
    
    \begin{figure}[!h]
        \centering
        \includegraphics[width=16cm]
            {../sources/game_ux/build/game_ux.png}
        \caption{Макет интерфейса}
    \end{figure}

    \subsection{Общая диаграмма с учётом каркаса}
    \subsection{Диаграмма последовательностей (sequence diagram)}
    \subsection{Диаграмма видов деятельности (activity diagram)}
    \newpage
    
    %
    \section{РАЗРАБОТКА АЛГОРИТМОВ ФУНКЦИОНИРОВАНИЯ И СТРУКТУР ДАННЫХ}
    \newpage
    
    %
    \section{РЕАЛИЗАЦИЯ ПРИЛОЖЕНИЯ И РЕЗУЛЬТАТЫ ИСПЫТАНИЙ}
    \subsection{Диаграмма компонентов (component diagram)}
    \subsection{Диаграмма развёртывания (deployment diagram)}
    \subsection{Тестирование приложения}
    \newpage

    %
    \newpage
    \addcontentsline{toc}{section}{СПИСОК ИСПОЛЬЗОВАННЫХ ИСТОЧНИКОВ}
    \section*{СПИСОК ИСПОЛЬЗОВАННЫХ ИСТОЧНИКОВ}
    \begin{enumerate}
        \item[1.] title- [Электронный ресурс]
        Режим доступа: \url{https://www.youtube.com/watch?v=}
        Дата~доступа:~xx.xx.2022.
    \end{enumerate}
    \newpage
\end{document}
