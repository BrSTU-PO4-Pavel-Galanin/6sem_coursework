\documentclass[12pt, a4paper, simple]{eskdtext}

\usepackage{hyperref}
\usepackage{env}
\usepackage{_sty/gpi_lst}
\usepackage{_sty/gpi_toc}
\usepackage{_sty/gpi_t}
\usepackage{_sty/gpi_p}
\usepackage{_sty/gpi_u}

% Код
\ESKDletter{}{К}{Р}
\def \gpiDocTypeNum {81}
\def \gpiDocVer {00}
\def \gpiCode {\ESKDtheLetterI\ESKDtheLetterII\ESKDtheLetterIII.\gpiStudentGroupName\gpiStudentGroupNum.\gpiStudentCard-0\gpiDocNum~\gpiDocTypeNum~\gpiDocVer}

\def \gpiDocTopic {ПОЯСНИТЕЛЬНАЯ ЗАПИСКА}

% Графа 1 (наименование изделия/документа)
\ESKDcolumnI {\ESKDfontIII \gpiTopic \\ \gpiDocTopic}

% Графа 2 (обозначение документа)
\ESKDsignature {\gpiCode}

% Графа 9 (наименование или различительный индекс предприятия) задает команда
\ESKDcolumnIX {\gpiDepartment}

% Графа 11 (фамилии лиц, подписывающих документ) задают команды
\ESKDcolumnXIfI {\gpiStudentSurname}
\ESKDcolumnXIfII {\gpiTeacherSurname}
\ESKDcolumnXIfV {\gpiTeacherSurname}

\begin{document}
    \input{_tex/gpi_pz_titlePage.tex}

    \ESKDthisStyle{empty}
    лист с заданием
    \newpage
    \ESKDthisStyle{formII}

    % Содержание
    \tableofcontents                                
    \paragraph{ПРИЛОЖЕНИЕ А. СХЕМА ПРОГРАММЫ}
    \paragraph{ПРИЛОЖЕНИЕ Б. ТЕКСТ ПРОГРАММЫ}
    \newpage

    %
    \newpage
    \addcontentsline{toc}{section}{ВВЕДЕНИЕ}
    \section*{ВВЕДЕНИЕ}

    Цель курсовой работы по дисциплине <<Совремменные платформы программирования>> - детальное проектирование и программная реализация сетевой игры.

    Клиенская часть написана на языке программирования Java Script, с использованием фреймворка ReactJS,
    с использованием объектно-ориентированного программирования.
    Серверная часть написана на языке программирования Node Java Script.

    Используя объектно-ориентированное проектирование позволяет обходить ряд сложных проблем в программировании,
    сводя необходимую модификацию программы к её расширению и дополнению.
    Систематическое применение объектно-ориентированного подхода позволяет разрабатывать достаточно хорошо структурированные,
    надежные в эксплуатации, просто модифицируемые программные системы.
    Элементы объектно-ориентированного программирования получили своё развитие, и в настоящее время ООП принадлежит к числу ведущих технологий программирования.

    \newpage

    %
    \section{АНАЛИЗ ПОСТАВКИ ЗАДАЧИ}
    \subsection{Перечень функций}

    Достижение цели курсовой работы предполагает необходимость создания веб- или мобильное приложение (ОС Android/iOS) со следующим функционалом:
    
    \begin{enumerate}
        \item[1.] Предусмотреть возможность игры с ИИ.
        \item[2.] Разработать фиксированный набор карт для игры (от 30 и более).
        \item[3.] Сохранение статистики матчей (опционально - сохранение на удаленном сервере).
        \item[4.] Поддержка сетевого матча один-на-один.
        \item[5.] Минимальный пользовательский интерфейс и графическое оформление карт.
    \end{enumerate}
    
    \subsection{Требования пользователей}

    Пользовательские требования - описание на естественном языке (плюс поясняющие диаграммы) функций, выполняемых системой, и ограничений, накладываемых на неё.

    Ответственный: Системный аналитик

    Эти требования должны определять только внешнее поведение системы, избегая по возможности определения структурных характеристик системы.
    Пользовательские требования должны быть написаны естественным языком с использованием простых таблиц, а также наглядных и понятных диаграмм.

    \textbf{Обязательные} требования игрока к игре:

    \begin{enumerate}
        \item [1.] Разработать фиксированный набор карт для игры (от 30 и более).
        \item [2.] Минимальный пользовательский интерфейс и графическое оформление карт.
    \end{enumerate}

    \textbf{Желательные} требования к игре:

    \begin{enumerate}
        \item [1.] Предусмотреть возможность игры с ИИ.
        \item [2.] Сохранение статистики матчей (опционально - сохранение на удаленном сервере).
        \item [3.] Поддержка сетевого матча один-на-один.
    \end{enumerate}

    \newpage
    \subsection{Описание предметной области}

    \textbf{Игрок} - человек или машина, который/которая играет в данную игру.
    В данной игре будут два игрока: мы и игрок противник.

    \textbf{Карты} - графический элемент игры, в виде прямоугольника с характеристиками.
    Карт в игре в количестве 30 штук. 30 штук как и у нас, так и у противника.
    Карты имеют фракции: мечник, лучник, тяжелая альтилерия; который сдавятся в определеное поле боя.
    
    \textbf{Мечник} - категория карты, которая имеет малый урон и сражается с мечниками противника.

    \textbf{Лучник} - категория карты, которая имеет малый урон и сражается с лучниками противника.

    \textbf{Тяжелая альтилерия} - катерия карты, которая имеет высокий урон и сражается с тяжелой альтилерией противника.

    \textbf{Поле боя} - игровое поле, которое разбито на 6 частей, из которых три части противника и три части наши.
    
    Поле боя содержит:

    \begin{enumerate}
        \item [1.] Тяжелую альтилерию противника.
        \item [2.] Лучников противника.
        \item [3.] Мечников противника.
        \item [4.] Своих мечников.
        \item [5.] Своих лучников.
        \item [6.] Свою тяжелую альтилерию.
    \end{enumerate}

    \textbf{ПАС} - действие, после объявления которого мы объявляем о прекращении бойни и пропускаем ход, в тоже время противнику дается последний выбор карты,
    после чего подсчитывается игровой счёт. В течении игры можна два раза нажать на кнопку <<ПАС>>.
    После подсчёта первого <<ПАС>> карты с поля боя идут на кладбище. Набираются карты до десяти.
    Происходит игра. После подсчёта второго <<ПАС>> сравниваются победы и объявляется победитель.

    \newpage
    \subsection{Варианты использования программы в виде диаграмм прецедентов (use case diagram)}

    Варианты использования программы изображены на рис.~\ref{fig:game__use_case_diagram}.

    \begin{figure}[!h]
        \centering
        \includegraphics[]
            {../sources/game_architecture/build/game__use_case_diagram.png}
        \caption{Первичное описание прецедентов}
        \label{fig:game__use_case_diagram}
    \end{figure}

    \paragraph{}\hspace{0pt}

    \textbf{\underline{Прецендент №1 <<Выбор карты>>}}

    \textbf{Назначение}: из своей колоды карт выбираем карту, которую ходим выставить на поле.

    \textbf{Исполнители}: игрок.

    \textbf{Предусловие}: есть карта/карты, которую/которые можно выбрать.

    \textbf{Основной поток событий}: игрок выберет карту и карты попадет на поле (мечника, лучника, тяжелой альтилерии),
    иначе АПС.

    \textbf{Альтернативный поток событий}: если у игрока не будет карт, то игрок пропускает ход;
    если игрок не выбирает карту, то по истечению времени карта выбирается автоматически.

    \paragraph{}\hspace{0pt}

    \textbf{\underline{Прецендент №2 <<Завершения сражения (ПАС)>>}}

    \textbf{Назначение}: завершить бой и узнать результат бойни.

    \textbf{Исполнители}: игрок.

    \textbf{Предусловие}: есть карты на поле.

    \textbf{Основной поток событий}: мы пропускаем ход, а противник выбирает последнюю карту, после чего проиходит подчёт сил, иначе АПС.

    \textbf{Альтернативный поток событий}: игрок может продолжать выставлять карты на поле.

    \newpage
    % \subsection{Первичное описание объектов и классов, прецедентов системы}
    \subsection{Первичное описание объектов и классов}

    Первичное описание класса игры изображено на рис.~\ref{fig:game__first_class_diagram}.

    \begin{figure}[!h]
        \centering
        \includegraphics[]
            {../sources/game_architecture/build/game__first_class_diagram.png}
        \caption{Первичное описание класса игры}
        \label{fig:game__first_class_diagram}
    \end{figure}

    Игра - класс содержащий логику игры.

    \begin{itemize}
        \item Колода карт противника и моя колода карт - начальный стак карточек,
        из которых в начале игры берется 10 рандомных карт.
        \item Карты на руках противника и мои карты на руках - карты, которые держим в руках,
        не могут превышать 10 штук. При количестве 0 штук - игрок пропускает ход.
        \item Кладбище карт противника и моё кладбище карт - карты которые ушли в отбой после первого сражения,
        так как игрок сделал ПАС и были подсчитаны результаты боя.
        \item Карты артилерии на поле противника и карты артилерии на моём поле - карты,
        которые выставлены на поле артилерии.
        Могут отсутствовать (0 штук).
        \item Карты лучников на поле противника и карты лучников на моём поле - карты,
        которые выставлены на поле лучников.
        Могут отсутствовать (0 штук).
        \item Карты мечников на поле противника и карты мечников на моём поле - карты,
        которые выставлены на поле мечника.
        Могут отсутствовать (0 штук).
    \end{itemize}

    % \subsection{Первоначаьное описание отношений между классами}

    \newpage
    \subsection{Диаграмма состояний (statechart diagram) для прецендентов}

    Диаграмма состояний для игры изображена на рис.~\ref{fig:game__statechart_diagram}.

    \begin{figure}[!h]
        \centering
        \includegraphics[]
            {../sources/game_architecture/build/game__statechart_diagram.png}
        \caption{Диаграмма состояний игры}
        \label{fig:game__statechart_diagram}
    \end{figure}

    \begin{itemize}
        \item Д1 - Начало игры (быбрали режим: играть с ИИ или играть с другим игроком по сети)
        \item Д2 - Выбираем карту из колоды, которой будет ходить (мечника, лучника, артилерию).
        \item Д3 - Заявляем о завершении боя: мы пропускаем ход, противник кладет последнюю карту, боя заканчивается.
        \item Д4 - Завершение первого боя и подсчёт сил. Старт второго боя
        \item Д5 - Страт второго боя.
        \item Д6 - Завершение второго боя и подсчёт сил. Сравниваются два боя. Объявляется победитель. Завершается игра.
    \end{itemize}

    %
    \newpage
    \section{ПРОЕКТИРОВАНИЕ СТРУКТУРЫ ПРИЛОЖЕНИЯ}

    % \subsection{Диаграммы классов предметной области}
    
    \subsection{Графический интерфейс приложения (UI/UX)}
    
    \begin{figure}[!h]
        \centering
        \includegraphics[width=12cm]
            {../sources/game_ux/build/game_ux.png}
        \caption{Макет интерфейса (UX)}
    \end{figure}

    \begin{figure}[!h]
        \centering
        \includegraphics[width=12cm]
            {../sources/game_ux/build/game_ui.png}
        \caption{Дизайн интерфейса (UI)}
    \end{figure}

    \begin{figure}[p!h]
        \centering
        \begin{minipage}{0.24\textwidth}
            \centering
            \includegraphics[height=6cm]
                {../sources/game_cards/build/swordsman_card_1.png}
            \caption{Урон 1 мечником}
            \label{fig:swordsman_card_1}
        \end{minipage}
        \begin{minipage}{0.24\textwidth}
            \centering
            \includegraphics[height=6cm]
                {../sources/game_cards/build/swordsman_card_2.png}
            \caption{Урон 2 мечником}
            \label{fig:swordsman_card_2}
        \end{minipage}
        \begin{minipage}{0.24\textwidth}
            \centering
            \includegraphics[height=6cm]
                {../sources/game_cards/build/swordsman_card_4.png}
            \caption{Урон 4 мечником}
            \label{fig:swordsman_card_4}
        \end{minipage}
        \begin{minipage}{0.24\textwidth}
            \centering
            \includegraphics[height=6cm]
                {../sources/game_cards/build/swordsman_card_5.png}
            \caption{Урон 5 мечником}
            \label{fig:swordsman_card_5}
        \end{minipage}
    \end{figure}

    \begin{figure}[p!h]
        \centering
        \begin{minipage}{0.24\textwidth}
            \centering
            \includegraphics[height=6cm]
                {../sources/game_cards/build/archer_card_4.png}
            \caption{Урон 1 лучником}
            \label{fig:artillery_card_1}
        \end{minipage}
        \begin{minipage}{0.24\textwidth}
            \centering
            \includegraphics[height=6cm]
                {../sources/game_cards/build/archer_card_5.png}
            \caption{Урон 5 лучником}
            \label{fig:artillery_card_5}
        \end{minipage}
        \begin{minipage}{0.24\textwidth}
            \centering
            \includegraphics[height=6cm]
                {../sources/game_cards/build/archer_card_6.png}
            \caption{Урон 6 лучником}
            \label{fig:artillery_card_6}
        \end{minipage}
    \end{figure}

    \begin{figure}[p!h]
        \centering
        \begin{minipage}{0.24\textwidth}
            \centering
            \includegraphics[height=6cm]
                {../sources/game_cards/build/artillery_card_1.png}
            \caption{Урон 1 артилерией}
            \label{fig:artillery_card_1}
        \end{minipage}
        \begin{minipage}{0.24\textwidth}
            \centering
            \includegraphics[height=6cm]
                {../sources/game_cards/build/artillery_card_5.png}
            \caption{Урон 5 артилерией}
            \label{fig:artillery_card_5}
        \end{minipage}
        \begin{minipage}{0.24\textwidth}
            \centering
            \includegraphics[height=6cm]
                {../sources/game_cards/build/artillery_card_6.png}
            \caption{Урон 6 артилерией}
            \label{fig:artillery_card_6}
        \end{minipage}
        \begin{minipage}{0.24\textwidth}
            \centering
            \includegraphics[height=6cm]
                {../sources/game_cards/build/artillery_card_8.png}
            \caption{Урон 8 артилерией}
            \label{fig:artillery_card_8}
        \end{minipage}
    \end{figure}

    \newpage
    \subsection{Общая диаграмма с учётом каркаса (layer diagram)}

    Общая диаграмма с учётом каркаса изображена на рис.~\ref{fig:game__layers_diagram}.

    \begin{figure}[!h]
        \centering
        \includegraphics[width=12cm]
            {../sources/game_architecture/build/game__layers_diagram.png}
        \caption{Общая диаграмма с учётом каркаса}
        \label{fig:game__layers_diagram}
    \end{figure}

    \subsection{Диаграмма последовательностей (sequence diagram)}

    Диаграмма последовательностей изображена на рис.~\ref{fig:game__layers_diagram}.

    \begin{figure}[!h]
        \centering
        \includegraphics[width=12cm]
            {../sources/game_architecture/build/game__sequence_diagram.png}
        \caption{Диаграмма последовательностей}
        \label{fig:game__sequence_diagram}
    \end{figure}
    
    \newpage
    \subsection{Диаграмма видов деятельности (activity diagram)}

    \begin{figure}[!h]
        \centering
        \includegraphics[]
            {../sources/game_architecture/build/game__activity_diagram.png}
        \caption{Диаграмма вида деятельности счётчика балов}
    \end{figure}

    \newpage
    
    %
    \section{РАЗРАБОТКА АЛГОРИТМОВ ФУНКЦИОНИРОВАНИЯ И СТРУКТУР ДАННЫХ}

    \paragraph{} \textbf{Алгоритм нахождения сумарного урона игрока}

    \underline{Исходные данные}: нет данных

    \underline{Алгоритм}:

    \begin{enumerate}
        \item[1.] counter = 0. Переход к пункту 2
        \item[2.] i = 0. Проходимся по массиву карточек карточек. Переход к пункту 2.1.
        \begin{enumerate}
            \item[2.1.] Если <<тип карточки>> равен <<на войне>>, то переход к пункту 2.2, иначе переход к пункту 2.3.
            \item[2.2.] counter = counter + <<урон карточки>>. Переход к пункту 2.3.
            \item[2.3.] i = i + 1. Если i < <<размер массива>>, то переход к пункту 2.1, иначе переход к пункту 3.
        \end{enumerate}
        \item[3.] return counter
    \end{enumerate}

    \paragraph{} \textbf{Алгоритм перетаскивания карточки на поле боя}

    \underline{Исходные данные}: 1) ИД карты

    \underline{Алгоритм}:

    \begin{enumerate}
        \item[1.] Если это не наш ход, то переход к пункту 12, иначе переход к пункту 2.
        \item[2.] Если мы выбрали ПАС, то переход к пункту 12, иначе переход к пункту 3.
        \item[3.] Если противник не ПАСовал, то переход к пункту 4, иначе переход к пункту 6.
        \item[4.] Блокируем доступ к нашим картам. Переход к пункту 5.
        \item[5.] Разблокируем доступ к картам противника. Переход к пункту 6.
        \item[6.] Перём карту под ИД равным id. Переход к пункту 7.
        \item[7.] Рендерим наши карты. Переход к пункту 8.
        \item[8.] Рендерим наш обший урон. Переход к пункту 9.
        \item[9.] Рендерим наш урон мечников. Переход к пункту 10.
        \item[10.] Рендерим наш урон лучников. Переход к пункту 11.
        \item[11.] Рендерим наш урон тяжолой артилерии. Переход к пункту 12.
        \item[12.] Конец.
    \end{enumerate}

    \paragraph{} \textbf{Алгоритм ПАСа у нас}

    \underline{Исходные данные}: нет данных

    \underline{Алгоритм}:

    \begin{enumerate}
        \item[1.] Если у нас есть ПАС, то переход к пункту 6, иначе переход к пункту 2.
        \item[2.] Устанавливаем значение true, что мы ПАСанули. Переход к пункту 3.
        \item[3.] Отлючаем доступ к нашим картам. Переход к пункту 4.
        \item[4.] Если противник не ПАСовал, то переход к пункту 5, иначе переход к пункту 6.
        \item[5.] Разблокировываем карты противника. Переход к пункту 6.
        \item[6.] Конец.
    \end{enumerate}

    \underline{Выходные данные}: мы пропустили ход, противник делает сколько угодно ходов, пока сам не ПАСёт.

    \paragraph{} \textbf{Алгоритм ПАСа у противника}

    \underline{Исходные данные}: нет данных

    \underline{Алгоритм}:

    \begin{enumerate}
        \item[1.] Если у противника есть ПАС, то переход к пункту 6, иначе переход к пункту 2.
        \item[2.] Устанавливаем значение true, что проивник ПАСанул. Переход к пункту 3.
        \item[3.] Отлючаем доступ к картам противника. Переход к пункту 4.
        \item[4.] Если мы не ПАСовал, то переход к пункту 5, иначе переход к пункту 6.
        \item[5.] Разблокировываем наши карты. Переход к пункту 6.
        \item[6.] Конец.
    \end{enumerate}

    \underline{Выходные данные}: противник пропустил ход, мы делаем сколько угодно ходов, пока мы не ПАСём.

    \paragraph{} Структура данных таблицы <<Результаты игры>>

    \lstinputlisting[language=sql, name=Создание таблицы <<Результаты игры>>]
        {../sources/database/initdb.d/GameResults.sql}

    \newpage
    
    %
    \section{РЕАЛИЗАЦИЯ ПРИЛОЖЕНИЯ И РЕЗУЛЬТАТЫ ИСПЫТАНИЙ}
    \subsection{Диаграмма компонентов (component diagram)}

    Диаграмма компонентов изображена на рис.~\ref{fig:game__component_diagram}.

    \begin{figure}[!h]
        \centering
        \includegraphics[width=16cm]
            {../sources/game_architecture/build/game__component_diagram.png}
        \caption{Диаграмма компонентов}
        \label{fig:game__component_diagram}
    \end{figure}

    % \newpage
    \subsection{Диаграмма развёртывания (deployment diagram)}

    Диаграмма развёртывания изображена на рис.~\ref{fig:game__deployment_diagram}.

    \begin{figure}[!h]
        \centering
        \includegraphics[width=16cm]
            {../sources/game_architecture/build/game__deployment_diagram.png}
        \caption{Диаграмма развёртывания}
        \label{fig:game__deployment_diagram}
    \end{figure}

    \newpage
    \subsection{Тестирование приложения}


    \subsubsection*{Тестирование просмотра своих карт в руках}

    \textit{Тестируемая задача}: \underline{Просмотр своих карт в руках}.
    
    \textit{Ожидаемый результат}: видны карты в руках.

    \textit{Полученный результат}:

    \textit{Вывод по тесту}:


    \subsubsection*{Тестирование добавление карты на стол}

    \textit{Тестируемая задача}: \underline{Добавление карты на стол}.
    
    \textit{Ожидаемый результат}: выбранная карта положилась на стол.

    \textit{Полученный результат}:

    \textit{Вывод по тесту}:


    \subsubsection*{Тестирование окончания боя}

    \textit{Тестируемая задача}: \underline{Окончание боя}.
    
    \textit{Ожидаемый результат}: мы пропускаем ход, а противник делает последний ход.
    
    \textit{Полученный результат}:

    \textit{Вывод по тесту}:

    \newpage

    %
    \newpage
    \addcontentsline{toc}{section}{СПИСОК ИСПОЛЬЗОВАННЫХ ИСТОЧНИКОВ}
    \section*{СПИСОК ИСПОЛЬЗОВАННЫХ ИСТОЧНИКОВ}
    \begin{enumerate}
        \item[1.] Получение GET и POST запросов на Node.js - YouTube
        - [Электронный ресурс]
        Режим доступа: \url{https://www.youtube.com/watch?v=YMJDUHUccvA}
        Дата~доступа:~19.11.2021.

        \item[2.] Подключение к базе данных MySQL в Node.js - YouTube
        - [Электронный ресурс]
        Режим доступа: \url{https://www.youtube.com/watch?v=YhuozY-qplI}
        Дата~доступа:~19.11.2021.

        \item[3.] Модули Node.js, require - YouTube
        - [Электронный ресурс]
        Режим доступа: \url{https://www.youtube.com/watch?v=1PkarXC-9TQ}
        Дата~доступа:~19.11.2021.

        \item[4.] Manual installation steps for older versions of WSL | Microsoft Docs
        - [Электронный ресурс]
        Режим доступа: \url{https://aka.ms/wsl2kernel}
        Дата~доступа:~19.11.2021.

        \item[5.] Леоненков А. В.
        Самоучитель UML 2. -- СПб.: БХВ-Петерберг, 2007. -- 576с.

        \item[6.] Почему в папке 'React-router-dom' нет экспорта Switch? — Хабр Q\&A
        - [Электронный ресурс]
        Режим доступа: \url{https://qna.habr.com/q/1103400}
        Дата~доступа:~15.05.2022.

        \item[7.] React Router | Upgrading from v5
        - [Электронный ресурс]
        Режим доступа: \url{https://reactrouter.com/docs/en/v6/upgrading/v5}
        Дата~доступа:~15.05.2022.

        \item[8.] reactjs - How to build a 404 page with react-router-dom v6 - Stack Overflow
        - [Электронный ресурс]
        Режим доступа: \url{https://stackoverflow.com/questions/67050966/how-to-build-a-404-page-with-react-router-dom-v6}
        Дата~доступа:~15.05.2022.

    \end{enumerate}
    \newpage
\end{document}
